

\documentclass[12pt]{article}

\usepackage[georgian ,english]{babel}
\usepackage[T8M]{fontenc}

\usepackage[]{fullpage}
\usepackage{amsmath}
\usepackage{physics}
\usepackage{mathtools}


\begin{document}

\title{ჯგუფთა წარმოდგენის თეორია}
\author{Andreas Niedens}
\date{\today}
\maketitle

\tableofcontents

\begin{sloppypar}
\section{აბსტრაქტი}
ჯგუფის ელემენტების წარმოდგენა შესაძლებელია მატრიცების სახით. ჩვან განვიხილავთ შესავალს წარმოდგენის თეორიაში. წარმოდგენის თეორია სავსეა მოულოდნელი და საინტერესო შედეგებით, როგორიცაა ორთოგონალურობის თერემა.

\section{შესავალი}

\subsection{განმარტებები}
ჯგუფთა თეორიის შესწავლის დაწყება შეიძლება სიმეტრიებით და გარდაქმნებით. წარმოდგენის თეორია თითო ჯგუფის ელემენტს შეუსაბამებს მატრიცას, რაც გვაძლევს საშუალებას, რომ ჯგუფის ელემენტებს ვუყუროთ როგორც სივრცეზე მოქმედ გარდაქმნებს. ეს არ არის ნებისმიერი მატრიცების სიმრავლე, მას აქვს მნიშვნელოვანი შეზღუდვა: მატრიცების გადამრავლებით (კომპიზიციით) ნარჩუნდება ჯგუფის სტრუქტურა. უფრო ფორმალურად,

\begin{equation} \label{Representation equation}
		D(g_1)D(g_2) = D(g_1 g_2)
\end{equation}
	
აქ წარმოდგენა $D$ თითო ჯგუფის ელემენტს $g$ შეუსაბამებს მატრიცას, ასე რომ ჩვენ დავწერთ ამ მატრიცას როგორც $D(g)$. რადგან მატრიცები უნდა იყოს შებრუნებადი, ისინი კვადრატული მატრიცები არიან განზომილებით $d \otimes d$, და $d$ ცნობილია როგორც წარმოდგენის განზომილება.

% TODO add more examples

ფიზიკოსები ხშირად გულისხმობენ მატრიცებს, როდესაც საუბრობენ ჯგუფის ელემენტებზე და ფიქრობენ, რომ ეს ერთი და იგივე არის. ბოლოს და ბოლოს, მათ მხოლოდ გარდაქმნები აღელვებთ. მაგალითად, როდესაც ფიზიკოსი ფიქრობს მობრუნებაზე, ის გულისხმობს $3 \otimes 3$ მატრიცას. მაგრამ არსებობს სხვანაირი ხედვაც, ძირითადად, მათემატიკოსებში, რომლებიც შეისწავლიან ჯგუფის ელემენტებს და მათ თვისებებს დამოუკიდებლად და არ ფიქრობენ, რომ ისინი სხვა ობიექტებზე მოქმედებენ. პრაქტიკაში, ბევრი ჯგუფი, რომელიც გამოიყენება თეორიულ ფიზიკაში, განიმარტება მატრიცების წარმოდგენით (მაგალითად, ლორენცის ჯგუფი).


შეგვიძლია ვნახოთ, რომ $D(I) = I_d$, სადაც მარცხენა $I$ გულისხმობს ჯგუფის ერთეულოვან ელემენტს, და ჩვენ გვაქვს $d \otimes d$ ერთეულოვანი მატრიცა მარჯვენა მხარეს. ზოგადად, $d$ არ არის ფიქსირებული რაიმე ჯგუფისთვის და დამოკიდებულია წარმოდგენაზე $D$. 

ამის საჩვენებლად, ჩვენ შევამოწმებთ, რომ 
$D(I)D(g) = D(Ig) = D(g)$. თუ ამ განტოლებას გავამრავლებთ $D(g^{-1})$-ზე მარჯვნიდან, ჩვენ მივიღებთ: $D(I)D(g)D(g^{-1}) = D(g)D(g^{-1})$, რომელიც, (\ref{Representation equation})-ის გათვალისწინებით, გვაძლევს $D(I)D(I) = D(I)$. გავამრავლოთ $(D(I))^{-1}$-ზე და მივიღებთ $D(I) = I_d$.
აქედან გამომდინარეობს $D(g^{-1}) = (D(g))^{-1}$, რადგან $D(g^{-1})$ და $D(g)$-ის გამრავლება გვაძლევს $I_d$-ს, რაც მოსალოდნელი იყო, რადგან ორი მატრიცა ერთმანეთს "აბათილებს", როგორც ჯგუფის ელემენტებს შეუძლიათ ერთმანეთის "გაბათილება".

ვცადოთ, უკეთესად გავერკვიოთ წარმოდგენებში. აქვს თუ არა ნებისმიერ ჯგუფს წარმოდგენა? რამდენი აქვს? არის თუ არა განზომილება შეზღუდული? როგორ გავარჩიოთ ორი წარმოდგენა, თუ მათ ერთნაირი განზომილება აქვთ?

ჩვენ ვიცით, რომ ყველა სასრული ჯგუფი $S_n$-ის რომელიმე ქვეჯგუფის იზომორფულია, თითო ელემენტი $S_n$-ში შეგვიძლია წარმოვიდგინოთ მატრიცული სახით (თითო ბაზისის ვექტორს შეგვიძლია შევუსაბამოთ რიცხვი და გადავანაცვლოთ ეს ვექტორები ერთმანეთში აბსტრაქტული ობიექტების გადანაცვლების მაგივრად). ასე რომ, ყველა სასრულ ჯგუფს გააჩნია წარმოდგენა. ზოგი ჯგუფის წარმოდგენა შეუძლებელია სასრულგანზომილებიანი მატრიცებით. ამის მიუხედავად, ჯგუფების ძირითად ნაწილს, რომელიც ფიზიკაში გვხვდება, გააჩნია სასრულგანზომილებიანი წარმოდგენა. ასე რომ წარმოდგენის თეორიის სწავლა მაინც სასარგებლოა.
 
არსებობს ტრივიალური წარმოდგენა : $D(g) = 1$, თუ განვიხილეთ 1 როგორც $1 \otimes 1$ მატრიცა. გასაგებია, რომ, (\ref{Representation equation}) სრულდება, რადგან ჩვენ ვამრავლებთ 1-იანებს მარჯვენა მხარეს. ამ წარმოდგენას ტრივიალური ეწოდება, მაგრამ, პირველი შეხედვით, მას გამოყენება არ გააჩნია. ამის მიუხედავად, მისი გამოყოფა მნიშვნელოვანია, რიცხვი 0-ს ანალოგიურად, რომელიც არ ცვლის არაფერს, თუ მას სხვა რიცხვს დავუმატებთ.

მათემატიკური ენის გამოყენებით, ჩვენ განვსაზღვრავთ $d$-განზომილებიან წარმოდგენას როგორც ასახვას ჯგუფიდან $G$ $GL(d, C)$-ის რომელიმე ქვესიმრავლეში. (\ref{Representation equation}) გვეუბნება, რომ ასახვა არის ჰომომორფული, მაგრამ თუ ის ასევე იზომორფულია, ან ერთი ერთში, ჩვენ ვამბობთ, რომ წარმოდგენა არის ერთგული.

ჩვენ ასევე შეგვიძლია მატრიცებით ჯგუფების განვსაზღვრა. ასეთ შემთხვევაში წარმოდგენას ეწოდება განმსაზღვრელი, ან ფუნდამენტური.

% TODO add examples

\subsection{ხასიათი კლასის ფუნქციაა}
ჯგუფებს შეიძლება ჰქონდეთ სხვადასხვა წარმოდგენები, ასე რომ ჩვენ მათ გავარჩევთ ზედა ინდექსის გამოყენებით. ჩვენ დავწერთ $D^{(r)}(g)$ მატრიცისთვის, რომელიც წარმოადგენს ჯგუფის ელემენტს $g$-ს წარმოდგენაში $r$.
	
	არსებობს ისეთი მნიშვნელოვანი ცნება, როგორიცაა ჯგუფის ხასიათი.

\begin{equation} \label{Group character definition}
		 \chi^{(r)}(g) \equiv tr D^{(r)}(g)
\end{equation}

ხასიათი, როგორც დასახელებიდან გამომდინარეობს, გვეხმარება წარმოდგენის დახასიათებაში.

ჩვეულებრივად, ხასიათი დამოკიდებულია წარმოდგენასა $r$ და ელემენტზე $g$. გავიხსენოთ, რომ ჯგუფი შეიძლება იყოს დაყოფილი ეკვივალენტურობის კლასებად. ორი ელემენტი $g_1$ და $g_2$ არის ეკვივალენტური ($g_1 \sim g_2$), თუ არსებოს სხვა ელემენტი $f$, რომ 

\begin{equation} \label{Group equivalence definition}
		 g_1 = f^{-1}g_2f
\end{equation}

ჩვენ შეგვიძლია ვნახოთ, რომ 

$
\chi^{(r)}(g_1) = tr D^{(r)}(g_1) = tr D^{(r)}(f^{-1}g_2f) = tr D^{(r)}(f^{-1}) D^{(r)}(g_2) D^{(r)}(f) = 
tr D^{(r)}(g_2) D^{(r)}(f) D^{(r)}(f^{-1}) = tr D^{(r)}(g_2) D^{(r)}(I) = tr D^{(r)}(g_2) = \chi^{(r)}(g_2)
$
სადაც ჩვენ გამოვიყენეთ განტოლება (\ref{Representation equation}) და კვალის ციკლურობის თვისება. სხვა სიტყვებით, თუ $g_1 \sim g_2$, მაშინ $\chi^{(r)}(g_2) = \chi^{(r)}(g_1)$. ასე რომ, თუ $(g \in c)$, მაშინ

\begin{equation} \label{Trace depends on class}
	\chi^{(r)}(c) = tr D^{(r)}(g) 
\end{equation}

სადაც $c$ აღნიშნავს ეკვივალენტურობის კლასს, რომელშიც არის ელემენტი $g$. მარჯვენა მხარეს მყოფი კვალი არ არის დამოკიდებული ელემენტზე $g$ იმდენად, როგორც დამოკიდებულია მის კლასსზეიმ კლასსზე, რომელშიც შედის ეს ელემენტი.

\subsection{ეკვივალენტური წარმოდგენები}
ორ წარმოდგენას $D(g)$ და $D'(g)$ ეწოდება ეკვივალენტური, თუ 

\begin{equation} \label{Equivalent representations definition}
	D'(g) = S^{-1}D'(g)S
\end{equation}

როგორც ვიცით წრფივი ალგებრიდან, $D(g)$ და $D'(g)$ არის ერთი და იგივე მატრიცა ჩაწერილი სხვადასხვა ბაზისში. ამ ორი ბაზისის ვექტორების ერთმანეთში გადაყვანა ხდება $S$ მარტიცის საშუალებით. სხვანაირად რომ ვთქვათ, თუ გვაქვს მოცემული წარმოდგენა $D(g)$, ჩვენ შეგვიძლია ავიღოთ შებრუნებადი მატრიცა $S$ და მისი გამოყენებით გავნმარტოთ $D'(g)$ ფორმულიდან (\ref{Equivalent representations definition}). მაშინ ისიც იქნება წარმოდგენა, რადგან 

$
D'(g_1)D'(g_2) = (S^{-1}D(g_1)S)(S^{-1}D(g_2)S) = S^{-1}D(g_1)D(g_2)S = S^{-1}D(g_1g_2)S = D'(g_1g_2)
$

აღვნიშნოთ, რომ ეს უნდა იყოს ერთი და იგივე $S$ ყველა $g$-სთვის. ზოგადად, მატრიცების სიმრავლე $D(g)$ და $D'(g)$ შეიძლება სხვადასხვანაირად გამოიყურებოდეს. თუ გვაქვს ორი წარმოდგენა, როგორ უნდა გავიგოთ, ეკვივალენტურები არიან თუ არა?

თუ ავიღებთ (\ref{Equivalent representations definition})-ს კვალს და გავიხსენებთ კვალის ციკლურობის თვისებას, მივიღებთ:
$
\chi'(c) = tr D'(g) = tr S^{-1}D(g)S = tr D(g)SS^{-1} = tr D(g) = \chi(c)
$

სადაც g არის c კლასის წევრი. აქედან გამომდინარეობს, რომ, თუ არსებობს რომელიმე კლასი ისეთი, რომ $\chi'(c) \neq \chi(c)$, მაშინ ორი წარმოდგენა არის განსხვავებული. მაგრამ თუ $\chi'(c) = \chi(c)$ სრულდება ყველა კლასისთვის, არის თუ არა ორი წარმოდგენა ერთნაირი? ზოგი თეორეთიკოსი ფიზიკოსისთვის ეს დებულება საკმარისია ეკვივალენტურობის დასამტკიცებლად, აბა დამთხვევა ხომ არ იქნება? ჩვენ ვნახავთ, რომ ეს მართლაც ასეა და ორი სხვადასხვა წარმოდგენისთვის 
$\chi^{r}(c)$ და $\chi^{s}(c)$ არის არა მხოლოდ განსხვავებული, არამედ რაღაც გაგებით ორთოგონალურიც კი.


\subsection{დაყვანადი და დაუყვანადი წარმოდგენა}
ახლა განვიხილავთ ასეთ მნიშვნელოვან ცნებას, როგორიცაა დაყვანადი და დაუყვანადი წარმოდგენა. მაგალითისთვის, განვიხილოთ ჯგუფი $SO(3)$. ჩვენ გვაქვს ტრივიალური 1-განზომილებიანი წარმოდგენა $D^{(1)}(g) = 1 $ და 3-განზომილებიანი განმსაზღვრელი წარმოდგენა $D^{(3)}(g)$. არსებობს თუ არა სხვა წარმოდგენები?

არსებობს თუ არა 8-განზომილებიანი წარმოდგენა? შეგვიძლია მოვიყვანოთ ასეთი მაგალითი: \\
$D(g) = $
$
\begin{pmatrix}
D^{(1)}(g) & 0 & 0 & 0\\
0 & D^{(1)}(g) & 0 & 0\\
0 & 0 & D^{(3)}(g) & 0\\
0 & 0 & 0 & D^{(3)}(g)
\end{pmatrix}
$ \\

ეს ნამდვილად არის $8\otimes 8$ მატრიცა, და თითო ელემენტს შეესაბამება ერთი ასეთი მატრიცა (8 იმიტომ რომ 1+ 1+ 3 + 3 = 8). ასეთ მატრიცას ეწოდაბა ბლოკური დიაგონალური მატრიცა. მასში ბლოკები დამოუკიდებლად მრავლდება როგორც მატრიცები, ანუ ბლოკებს ერთმანეთზე არ აქვთ არანაირი გავლენა გამრავლებისას. ასევე უნდა აღვნიშნოთ, რომ ამ მატრიცაში სიმბოლო 0 შეიძლება აღნიშნავდეს $1\otimes 1$, $3\otimes 1$, $1\otimes 3$ ან $3\otimes 3$ მატრიცას, რომელშიც ყველა მნიშვნელობა არის 0-ს ტოლი, რადგან $D^{(3)}(g)$ და $D^{(1)}(g)$ -ს განზომილებები არ ემთხვევა ერთმანეთს. ასეთი აღნიშვნა შემოვიღეთ თვალსაჩინოებისთვის. \\

ჩვენ მივადგით ორი $D^{(1)}(g)$ და ორი $D^{(3)}(g)$ ერთად, და ვამბობთ, რომ მივიღეთ ახალი წარმოდგენა. ასეთი იაფი ტრიუკით ჩვენ შეგვიძლია ნებისმიერი განზომილების წარმოდგენა მივიღოთ (ყველაზე მარტივად, რამდენიც გვინდა $D^{(1)}(g)$ მივადგათ ერთმანეთს. გამოვა ერთეულოვანი მატრიცა, ანუ ჯგუფი ინვარიანტულს ტოვებს სივრცეს.) როგორც რაციონალური ხალხი, ჩვენ უნდა შევთანხმდეთ, რომ ასეთი წარმოდგენა "ახალ" წარმოდგენას არ გვაძლევს. \\

წარმოდგენა $D(g)$-ს ეწოდება დაყვანადი,და ის ჩაიწერება როგორც პირდაპირი ჯამი იმ წარმოდგენების, რომლებზეც ის დაიყვანება. ჩვენს მაგალითში, $D(g) = D^{(1)}(g)  \oplus  D^{(1)}(g)  \oplus D^{(3)}(g)   \oplus D^{(3)}(g)$.\\

წარმოდგენებს, რომლებიც არ არიან დაყვანადი, დაუყვანადი ეწოდებათ. გასაგებია, რომ ჩვენ უნდა შევისწავლოთ დაუყვანადი წარმოდგენები. \\

ნათელია, რომ $D(g)$ არის დაყვანადი, მაგრამ თუ ვინმე გაამრავლებს ამ მატრიცას $S$-ზე ისე, რომ მიიღებს $D'(g) = S^{-1}D'(g)S$, მაშინ გაგვიჭირდება თქმა, არის თუ არა $D'(g)$ დაუყვანადი, რადგან, ზოგადად, მას აღარ ექნება ბლოკური დიაგონალური ფორმა და არც 0-ებით სავსე დარჩება.\\

გავიხსენოთ ტრივიალური წარმოდგენის განმარტება $D^{(1)}(g) = 1$. რატომ გამოვიყენეთ 1, და არა ერთეულოვანი $I_k$ მატრიცა განზომილებით $k\otimes k$ რომელიმე დადებითი $k$ რიცხვისთვის? საქმე იმაშია, რომ ასეთი მატრიცა იქნებოდა დაყვანადი თუ $k \neq 1$. წარმოდგენა $D^{(1)}(g)$, შეიძლება, ტრივიალურია, მაგრამ დაყვანადი არ არის.\\

წარმოდგენის თეორიის ერთ-ერთი მიზანია, გავარკვიოთ რაიმე კრიტერიუმით, არის თუ არა წარმოდგენა დაუყვანადი და ჩამოვთვალოთ ყველა დაუყვანადი წარმოდგენა. რადგან ყველა ჯგუფს აქვს უსასრულო რაოდენობის დაყვანადი წარმოდგენა, ჩვენ გვაინტერესებს, რამდენი დაუყვანადი წარმოდგენა აქვს მას.

\subsection{შეზღუდვა ქვეჯგუფზე}

$G$ ჯგუფის წარმოდგენა ასევე გვაძლევს საშუალებას, გვქონდეს წარმოდგენა რომელიმე მისი ქვეჯგუფისთვის. აღვნიშნოთ $H$ ქვეჯგუფის ელემენტები, როგორც $h$. თუ $D(g_1)D(g_2) = D(g_1 g_2)$ ნებისმიერი ჯგუფის ელემენტისთვის, მაშინ იგივე სრულდება ჯგუფის ქვეჯგუფის ელემენტებისთვის : $D(h_1)D(h_2) = D(h_1 h_2)$. ასეთ $H$-ის წარმოდგენას ეწოდება $G$-ს წარმოდგენა, რომელიც არის შეზღუდული $H$-ზე.\\

როდესაც წარმოდგენა არის შეზღუდული ქვეჯგუფზე, ზოგადად, $G$-ში დაუყვანადი წარმოდგენა შეიძლება აღარ იყოს დაუყვანადი $H$-ში. დიდი ალბათობით, ის დაიშლება $H$-ის დაუყვანად წარმოდგენებად. ამის მიზეზი გასაგებია: შეიძლება არსებობდეს რამე ბაზისი, რომელშიც $D(h)$-ს აქვს ბლოკური დიაგონალური ფორმა ყველა $h$ - ისთვის, მაგრამ ეს შეიძლება არ იყოს სამართლიანი დანარჩენი $g$ - სთვის, რომელიც არ შედის $H$-ში. უფრო მარტივად რომ ვთქათ, $H$-ში არის უფრო ცოტა ელემენტი.\\

ის, თუ როგორ იშლება $G$-ს დაუყვანადი წარმოდგენა $H$-ში, ერთ-ერთი ყველაზე მნიშვნელოვანი საკითხია.

\subsection{უნიტარული წარმოდგენები}

აქ განვიხილავთ საკვანძო თეორემას, რომელიც გვეუბნება, რომ ყველა სასრულ ჯგუფს აქვს უნიტარული წარმოდგენა, ანუ $D(g)^{\dagger}D(g)=I$ ყოველი $g$-სა და წარმოდგენისთვის.

პრაქტიკაში, ეს თეორემა გვეხმარება სასრული ჯგუფების წარმოდგენების მოძიებაში. დასაწყისისთვის, ჩვენ შეგვიძლია უგულვებელყოთ ნებისმიერი მატრიცა, რომელიც არ არის უნიტარული.

მოდით, ინტუიციურად დავინახოთ, რას გულისხმობს ეს თეორემა. წარმოვიდგინოთ, რომ წარმოდგენა $D(g)$ არის $1\otimes 1$, ანუ კომპლექსური რიცხვი $re^{i\theta}$. ჩვენ ვიცით, რომ, თუ ავიღეთ სასრული ჯგუფის რომელიმე ელემენტი $g$ და ის თავის თავზე ვამრავლეთ, მაშინ სასრული რაოდენობა მოქმედებების შედეგად ჩვენ ერთეულოვან ელემენტს მივიღებთ : $g^k=I$ რომელიმე $k$ -სთვის. მაგრამ $g^k$ არის წარმოდგენილი, როგორც $D(g^k)= D(g)^k = r^ke^{ik\theta}$. თუ $r \neq 1$, მაშინ ეს ელემენტი ვერ იქნება $I$. ამიტომაც, $D(g)$ უნიტარულია.
 
\subsection{უნიტარულობის თეორემის დამტკიცება}

გადანაცვლების ლემის თანახმად, თუ მოცემული გვაქვს ფუნქცია ჯგუფის ელემენტებზე, მაშინ, ყოველი $g' \in G$-სთვის:

\begin{equation} \label{Rearrangement lemma}
		\sum_{g \in G}f(g) = \sum_{g \in G}f(g'g) = \sum_{g \in G}f(gg')
\end{equation}

ეს სამი ჯამი სინამდვილეში ერთნაირი, მაგრამ გადანაცვლებული წევრებისგან შედგება, ამიტომაც ტოლობა არის სამართლიანი.\\

ახლა დავამტკიცოთ უნიტარულობის თეორემა. 

დავუშვათ, რომ მოცემული წარმოდგენა $\widetilde D(g)$ არ არის უნიტარული. შემოვიღოთ განმარტება 

$$H = \sum_{g} \widetilde D(g)^{\dagger} \widetilde D(g) $$

\sloppy სადაც ჯამში ვგულისხმობთ ყველა ელემენტს $G$-დან. მაშინ ნებისმიერი ელემენტი $g'$-სთვის: 
$
 \widetilde D(g')^{\dagger} H \widetilde D(g') = 
 \sum_{g} \widetilde D(g')^{\dagger} \widetilde D(g)^{\dagger} \widetilde D(g) \widetilde D(g') = 
 \sum_{g} (\widetilde D(g) \widetilde D(g') )^{\dagger} \widetilde D(
 g) \widetilde D(g') =
 \sum_{g} (\widetilde D(gg') )^{\dagger} \widetilde D(gg') = H
$

ბოლო ტოლობა სამართლიანია გადანაცვლების ლემის გამო. მარტიცა $H$ არის "ინვარიანტული".
რადგან $H$ ერმიტულია, არსებობს უნიტარული მატრიცა $W$ ისეთი, რომ $\rho^2=W^{\dagger}HW$ არის ნამდვილი და დიაგონალური. ახლა ვაჩვენოთ, რომ დიაგონალური ელემენტები არა მარტო რეალურია, არამედ დადებითიც. (ამიტომაც შემოვიღეთ აღნიშვნა $\rho^2$: ჩვენ შეგვიძლია მისი ფესვის ამოღება, რომ მივიღოთ დიაგონალური და ნამდვილი მატრიცა $\rho$). ამისთვის გამოვიყენოთ თეორემა, რომელიც გვეუბნება, რომ ნებისმიერი მატრიცისთვის $M$, მატრიცა $M^{\dagger}M$-ს აქვს არაუარყოფითი საკუთვრივი რიცხვები. დავუშვათ, $\psi$ იყოს სვეტი ვექტორი, რომელსაც აქვს 1 j-ურ პოზიციაზე, სხვაგან კი 0 არის.

$
 (\rho^2)_{jj} = 
 \psi^{\dagger} \rho^2 \psi =
 \psi^{\dagger} W^{\dagger}HW \psi = 
 \sum_{g} \psi^{\dagger} W^{\dagger}  \widetilde D(g)^{\dagger} \widetilde D(g) W \psi = 
 \sum_{g} \phi(g)^{\dagger} \phi(g) > 0
$

% TODO explain this inequality better

(აქ ჩვენ განვმარტეთ $\phi(g) = \widetilde D(g) W \psi$). გამოდის, მატრიცა $\rho$ არსებობს და მას აქვს ზემოაღნიშნული თვისებები.\\

შემოვიღოთ კიდევ ერთი აღნიშვნა: $D(g) \equiv \rho W^{\dagger} \widetilde D(g) W \rho^{-1}$. ვაჩვენოთ, რომ $D(g)$ არის უნიტარული. ამისთვის გამოვთვალოთ 
$D(g)^{\dagger}= \rho^{-1} W^{\dagger} \widetilde D(g)^{\dagger} W \rho$
, ასე რომ

$
 D(g)^{\dagger}D(g) = 
 \rho^{-1} W^{\dagger} \widetilde D(g)^{\dagger} W \rho^2 
 W^{\dagger} \widetilde D(g) W \rho^{-1} = 
 \rho^{-1} W^{\dagger} \widetilde D(g)^{\dagger} H \widetilde D(g) W \rho^{-1} = 
 \rho^{-1} W^{\dagger} H W \rho^{-1} = 
 \rho^{-1} \rho^{2} \rho^{-1} = 
 I
$

მესამე ტოლობა სრულდება, რადგან $H = \widetilde D(g')^{\dagger} H \widetilde D(g')$, როგორც უკვე ვაჩვენეთ. \\

უნიტარულობის თეორემა დამტკიცებულია. საინტერესოა, რას მივიღებდით, თუკი $\widetilde D(g)$ თავიდანვე იქნებოდა უნიტარული. მაშინ $H = N(G)I$, სადაც $N(G)$ არის ჯგუფში ელემენტების რაოდენობა, ასე რომ $W = I$, $\rho = \sqrt{N(G)}I$ არის ერთეულოვანი მატრიცის პროპორციული, ასე რომ $D(g) = \rho \widetilde D(g) \rho^{-1} = \widetilde D(g)$.

ბევრ მაგალითში წარმოდგენის მატრიცა არის ნამდვილი და არა კომპლექსური. ნამდვილი უნიტარული მატრიცა ორთოგონალურია, სხვა სიტყვებით, თუ $D(g)$ ნამდვილია, მაშინ $D(g)^TD(g) = I$ ყველა $g$-სთვის.

\subsection{ნამრავლის წარმოდგენა}
წარმოდგენების ერთმანეთზე გადაბმა არის მარტივი, მაგრამ უინტერესო გზა, მივიღოთ უფრო დიდი წარმოდგენა, რომელსაც ჩვენ ვუწოდებთ პარატა წარმოდგენებიდან პირდაპირი ნამრავლით მიღებულ წარმოდგენას.\\

თუ მოცემული გვაქვს ორი წარმოდგენა განზომილებებით $d_r$ და $d_s$, და წარმოდგენის მატრიცებით $D^{(r)}(g)$ და $D^{(s)}(g)$, ჩვენ შეგვიძლია განვმარტოთ წარმოდგენა, რომელიც არის განსაზღვრული მატრიცების პირდაპირი ნამრავლით $D(g) = D^{(r)}(g) \otimes D^{(s)}(g)$, ანუ $d_{r}d_{s} * d_{r}d_{s}$ -ზე მატრიცით, რომელიც მოცემულია, როგორც
$$D(g)_{a\alpha,b\beta} = D^{(r)}(g)_{ab} D^{(s)}(g)_{\alpha\beta}$$

აქ ჩვენ სპეციალურად გამოვიყენეთ სხვადასხვა ასოები $D^{(r)}(g)$-სა და $D^{(s)}(g)$-ზე იმის აღსანიშნავად, რომ ეს სხვადასხვანაირი ობიექტებია და ინდექსები სხვადასხვა ინტერვალებს გარბიან, კერძოდ: $a$, $b$ $=1,\dots d_r$, და $\alpha$, $\beta$ $=1,\dots d_s$.

პირდაპირი ნამრავლის მატრიცების გადამრავლების წესია: \\
$
 D(g)D(g') = (D^{(r)}(g) \otimes D^{(s)}(g))(D^{(r)}(g') \otimes D^{(s)}(g')) =
 D^{(r)}(g)D^{(r)}(g') \otimes D^{(s)}(g)D^{(s)}(g') = D(gg')
$

მართლაც გვეუბნება, რომ ნამრავლის წარმოდგენა მართლაც არის წარმოდგენა. ზოგადად, არ არის მიზეზი, რომ ეს ნამრავლი დაუყვანადი იყოს. ჩვენ უნდა ვისწავლოთ, როგორ დაიყვანება ნამრავლის წარმოდგენა დაუყვანადი წარმოდგენების პირდაპირ ნამრავლზე.\\

პირდაპირი ნამრავლის ხასიათი შეიძლება იყოს მარტივად გამოთვლილი, თუ ჩვენ ინდექსს $a\alpha$ გავუტოლებთ $b\beta$ და ავჯამავთ:

$
\chi(c) = \sum_{a,\alpha}D(g)_{a\alpha,a\alpha}= (\sum_aD^{(r)}(g)_{aa})(\sum_{\alpha}D^{(s)}(g)_{\alpha\alpha}) = \chi^{(r)}(c)\chi^{(s)}(c)
$
ჩვეულებრივად, $c$ აღნიშნავს კლასს, რომელსაც ეკუთვნის ჯგუფის ელემენტი $g$. პირდაპირი ნამრავლის წარმოდგენის ხასიათი არის წარმოდგენების ხასიათების ნამრავლი, რომლებისგანაც ის შედგება. ეს შედეგი კარგ ანალოგიაშია წინასთან, როდესაც ჩვენ მივიღეთ, რომ პირდაპირი ჯამის წარმოდგენის ხასიათი არის წარმოდგენების ხასიათების ჯამი, რომლებისგანაც ის შედგება.

ამ მსჯელობაში ჩვენ არ მოგვითხოვია, რომ $r$ და $s$ იყვნენ დაუყვანადი.\\

ფიზიკაში, ხშირად გამოსადეგია ვიფიქროთ ობიექტებზე, რომლებზეც მოქმედებს ჯგუფი. დავუშვათ, $\phi_a$, $a = 1,\dots,d_r$, აღნიშნავდეს $d_r$ ობიექტს, რომლებიც გარდაიქმნებიან ერთმანეთის წრფივ კომბინაციებში. ანალოგიურად, $\xi_{\alpha}$, $\alpha = 1,\dots,d_s$ აღვნიშნოთ ობიექტები $s$ წარმოდგენისთვის. მაშინ $d_sd_r$ ობიექტი $\phi_a\xi_{\alpha}$ განსაზღვრავს პირდაპირი ნამრავლის წარმოდგენას $r \otimes s$. როგორც ვნახავთ, ეს აბსტრაქტული მათემატიკური ობიექტები არის რეალიზებული კვანტურ მექანიკაში ტალღური ფუნქციების სახით.








\end{sloppypar}
\end{document}







