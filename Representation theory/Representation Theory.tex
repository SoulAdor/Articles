

\documentclass[12pt]{article}

\usepackage[georgian ,english]{babel}
\usepackage[T8M]{fontenc}

\usepackage[]{fullpage}
\usepackage{amsmath}
\usepackage{physics}
\usepackage{mathtools}


\begin{document}

\title{ჯგუფთა წარმოდგენის თეორია}
\author{Andreas Niedens}
\date{\today}
\maketitle

\tableofcontents

\section{აბსტრაქტი}
ჯგუფის ელემენტების წარმოდგენა შესაძლებელია მატრიცების სახით. ჩვან განვიხილავთ შესავალს წარმოდგენის თეორიაში. წარმოდგენის თეორია სავსეა მოულოდნელი და საინტერესო შედეგებით, როგორიცაა ორთოგონალურობის თერემა.

\section{შესავალი}

\subsection{განმარტებები}
ჯგუფთა თეორიის შესწავლის დაწყება შეიძლება სიმეტრიებით და გარდაქმნებით. წარმოდგენის თეორია თითო ჯგუფის ელემენტს შეუსაბამებს მატრიცას, რაც გვაძლევს საშუალებას, რომ ჯგუფის ელემენტებს ვუყუროთ როგორც სივრცეზე მოქმედ გარდაქმნებს. ეს არ არის ნებისმიერი მატრიცების სიმრავლე, მას აქვს მნიშვნელოვანი შეზღუდვა: მატრიცების გადამრავლებით (კომპიზიციით) ნარჩუნდება ჯგუფის სტრუქტურა. უფრო ფორმალურად,

\begin{equation} \label{Representation equation}
		D(g_1)D(g_2) = D(g_1 g_2)
\end{equation}
	
აქ წარმოდგენა $D$ თითო ჯგუფის ელემენტს $g$ შეუსაბამებს მატრიცას, ასე რომ ჩვენ დავწერთ ამ მატრიცას როგორც $D(g)$. რადგან მატრიცები უნდა იყოს შებრუნებადი, ისინი კვადრატული მატრიცები არიან განზომილებით $d \otimes d$, და $d$ ცნობილია როგორც წარმოდგენის განზომილება.

% TODO add more examples

ფიზიკოსები ხშირად გულისხმობენ მატრიცებს, როდესაც საუბრობენ ჯგუფის ელემენტებზე და ფიქრობენ, რომ ეს ერთი და იგივე არის. ბოლოს და ბოლოს, მათ მხოლოდ გარდაქმნები აღელვებთ. მაგალითად, როდესაც ფიზიკოსი ფიქრობს მობრუნებაზე, ის გულისხმობს $3 \otimes 3$ მატრიცას. მაგრამ არსებობს სხვანაირი ხედვაც, ძირითადად, მათემატიკოსებში, რომლებიც შეისწავლიან ჯგუფის ელემენტებს და მათ თვისებებს დამოუკიდებლად და არ ფიქრობენ, რომ ისინი სხვა ობიექტებზე მოქმედებენ. პრაქტიკაში, ბევრი ჯგუფი, რომელიც გამოიყენება თეორიულ ფიზიკაში, განიმარტება მატრიცების წარმოდგენით (მაგალითად, ლორენცის ჯგუფი).


შეგვიძლია ვნახოთ, რომ $D(I) = I_d$, სადაც მარცხენა $I$ გულისხმობს ჯგუფის ერთეულოვან ელემენტს, და ჩვენ გვაქვს $d \otimes d$ ერთეულოვანი მატრიცა მარჯვენა მხარეს. ზოგადად, $d$ არ არის ფიქსირებული რაიმე ჯგუფისთვის და დამოკიდებულია წარმოდგენაზე $D$. 

ამის საჩვენებლად, ჩვენ შევამოწმებთ, რომ 
$D(I)D(g) = D(Ig) = D(g)$. თუ ამ განტოლებას გავამრავლებთ $D(g^{-1})$-ზე მარჯვნიდან, ჩვენ მივიღებთ: $D(I)D(g)D(g^{-1}) = D(g)D(g^{-1})$, რომელიც, (\ref{Representation equation})-ის გათვალისწინებით, გვაძლევს $D(I)D(I) = D(I)$. გავამრავლოთ $(D(I))^{-1}$-ზე და მივიღებთ $D(I) = I_d$.
აქედან გამომდინარეობს $D(g^{-1}) = (D(g))^{-1}$, რადგან $D(g^{-1})$ და $D(g)$-ის გამრავლება გვაძლევს $I_d$-ს, რაც მოსალოდნელი იყო, რადგან ორი მატრიცა ერთმანეთს "აბათილებს", როგორც ჯგუფის ელემენტებს შეუძლიათ ერთმანეთის "გაბათილება".

ვცადოთ, უკეთესად გავერკვიოთ წარმოდგენებში. აქვს თუ არა ნებისმიერ ჯგუფს წარმოდგენა? რამდენი აქვს? არის თუ არა განზომილება შეზღუდული? როგორ გავარჩიოთ ორი წარმოდგენა, თუ მათ ერთნაირი განზომილება აქვთ?

ჩვენ ვიცით, რომ ყველა სასრული ჯგუფი $S_n$-ის რომელიმე ქვეჯგუფის იზომორფულია, თითო ელემენტი $S_n$-ში შეგვიძლია წარმოვიდგინოთ მატრიცული სახით (თითო ბაზისის ვექტორს შეგვიძლია შევუსაბამოთ რიცხვი და გადავანაცვლოთ ეს ვექტორები ერთმანეთში აბსტრაქტული ობიექტების გადანაცვლების მაგივრად). ასე რომ, ყველა სასრულ ჯგუფს გააჩნია წარმოდგენა. ზოგი ჯგუფის წარმოდგენა შეუძლებელია სასრულგანზომილებიანი მატრიცებით. ამის მიუხედავად, ჯგუფების ძირითად ნაწილს, რომელიც ფიზიკაში გვხვდება, გააჩნია სასრულგანზომილებიანი წარმოდგენა. ასე რომ წარმოდგენის თეორიის სწავლა მაინც სასარგებლოა.
 
არსებობს ტრივიალური წარმოდგენა : $D(g) = 1$, თუ განვიხილეთ 1 როგორც $1 \otimes 1$ მატრიცა. გასაგებია, რომ, (\ref{Representation equation}) სრულდება, რადგან ჩვენ ვამრავლებთ 1-იანებს მარჯვენა მხარეს. ამ წარმოდგენას ტრივიალური ეწოდება, მაგრამ, პირველი შეხედვით, მას გამოყენება არ გააჩნია. ამის მიუხედავად, მისი გამოყოფა მნიშვნელოვანია, რიცხვი 0-ს ანალოგიურად, რომელიც არ ცვლის არაფერს, თუ მას სხვა რიცხვს დავუმატებთ.

მათემატიკური ენის გამოყენებით, ჩვენ განვსაზღვრავთ $d$-განზომილებიან წარმოდგენას როგორც ასახვას ჯგუფიდან $G$ $GL(d, C)$-ის რომელიმე ქვესიმრავლეში. (\ref{Representation equation}) გვეუბნება, რომ ასახვა არის ჰომომორფული, მაგრამ თუ ის ასევე იზომორფულია, ან ერთი ერთში, ჩვენ ვამბობთ, რომ წარმოდგენა არის ერთგული.

ჩვენ ასევე შეგვიძლია მატრიცებით ჯგუფების განვსაზღვრა. ასეთ შემთხვევაში წარმოდგენას ეწოდება განმსაზღვრელი, ან ფუნდამენტური.

% TODO add examples

\subsection{ხასიათი კლასის ფუნქციაა}
ჯგუფებს შეიძლება ჰქონდეთ სხვადასხვა წარმოდგენები, ასე რომ ჩვენ მათ გავარჩევთ ზედა ინდექსის გამოყენებით. ჩვენ დავწერთ $D^{(r)}(g)$ მატრიცისთვის, რომელიც წარმოადგენს ჯგუფის ელემენტს $g$-ს წარმოდგენაში $r$.
	
	არსებობს ისეთი მნიშვნელოვანი ცნება, როგორიცაა ჯგუფის ხასიათი.

\begin{equation} \label{Group character definition}
		 \chi^{(r)}(g) \equiv tr D^{(r)}(g)
\end{equation}

ხასიათი, როგორც დასახელებიდან გამომდინარეობს, გვეხმარება წარმოდგენის დახასიათებაში.

ჩვეულებრივად, ხასიათი დამოკიდებულია წარმოდგენასა $r$ და ელემენტზე $g$. გავიხსენოთ, რომ ჯგუფი შეიძლება იყოს დაყოფილი ეკვივალენტურობის კლასებად. ორი ელემენტი $g_1$ და $g_2$ არის ეკვივალენტური ($g_1 \sim g_2$), თუ არსებოს სხვა ელემენტი $f$, რომ 

\begin{equation} \label{Group equivalence definition}
		 g_1 = f^{-1}g_2f
\end{equation}

ჩვენ შეგვიძლია ვნახოთ, რომ 

$
\chi^{(r)}(g_1) = tr D^{(r)}(g_1) = tr D^{(r)}(f^{-1}g_2f) = tr D^{(r)}(f^{-1}) D^{(r)}(g_2) D^{(r)}(f) = 
tr D^{(r)}(g_2) D^{(r)}(f) D^{(r)}(f^{-1}) = tr D^{(r)}(g_2) D^{(r)}(I) = tr D^{(r)}(g_2) = \chi^{(r)}(g_2)
$
სადაც ჩვენ გამოვიყენეთ განტოლება (\ref{Representation equation}) და კვალის ციკლურობის თვისება. სხვა სიტყვებით, თუ $g_1 \sim g_2$, მაშინ $\chi^{(r)}(g_2) = \chi^{(r)}(g_1)$. ასე რომ, თუ $(g \in c)$, მაშინ

\begin{equation} \label{Trace depends on class}
	\chi^{(r)}(c) = tr D^{(r)}(g) 
\end{equation}

სადაც $c$ აღნიშნავს ეკვივალენტურობის კლასს, რომელშიც არის ელემენტი $g$. მარჯვენა მხარეს მყოფი კვალი არ არის დამოკიდებული ელემენტზე $g$ იმდენად, როგორც დამოკიდებულია მის კლასსზეიმ კლასსზე, რომელშიც შედის ეს ელემენტი.


\end{document}







